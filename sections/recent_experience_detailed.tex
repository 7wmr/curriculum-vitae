
\edef\hc{\string:}

% Contino, Principal Consultant
\cventry % chktex 1
  {\mbox{September 2020 -} \mbox{Present}}
  {Principal Consultant}
  {Contino}{London}{}
  { 
    ...
  }

% Contino, Senior Consultant
\cventry % chktex 1
  {\mbox{October 2019 -} \mbox{September 2020}}
  {Senior Consultant}
  {Contino}{London}{}
  {
    Retail Banking Client
    \begin{itemize}
      \item Cloud 
    \end{itemize}
  }

% IPsoft, Automation Manager, Europe
\cventry % chktex 1
  {\mbox{November 2018 -} \mbox{October 2019}}
  {Automation Team Manager, Europe}
  {IPsoft}{London}{}
  { 
    Manager of the Automation team in Europe, responsible for the 
    development \& delivery of automations on IPcenter: an IT service management 
    platform; as well as 1Desk: the next generation autonomic tool.\\~\\
    Led the project assigned by European CEO to containerise (Docker) our IPcenter 
    IT service management platform product using Kubernetes (Helm) within a 2-month timeframe. 
    The objectives achieved were to focus on the automation and related services, 
    producing a lightweight, rapidly deployable automation engine. 
    This product could be provisioned in AWS within minutes, including any client 
    customisation and pre-built automations. 
    Also included was an integration with Elastic Search to provide real-time 
    reporting of automation executions via Kibana dashboards.\\~\\
    Proposed and implemented the automation standardisation program to ensure all 
    automations conformed to the same documented development and testing standards. 
    This included the use of Python Sphinx / reStructuredText as well as the development 
    of automated QA reporting for automation engineers. 
    This has resulted in automations that are easily portable between instances, 
    as well as ensuring that engineers can support multiple clients effectively (without retraining).\\ 
  }

% IPsoft, Automation Manager, United Kingdom
\cventry % chktex 1
  {\mbox{November 2016 -} \mbox{November 2018}}
  {Automation Team Lead, United Kingdom}
  {IPsoft}{London}{}
  {
    Leading the automation team in the UK primarily working with our new client (a leading UK bank) to 
    configure the IPcenter AaaS Instance to their needs. This included developing custom 
    automations to execute on and remediate 10K+ Windows events from their monitoring platforms.\\~\\ 
    Also, I designed and implemented integrations with client CMDB data, password stores 
    and incident/change management (ServiceNow), using SQL and RESTful APIs. Since 
    implementation, the engineer time saved (as calculated by the client) has risen to approximately 
    10K hours per month.\\
  }

% IPsoft, Senior Automation Engineer
\cventry % chktex 1
  {\mbox{June 2016 -} \mbox{November 2016}}
  {Senior Automation Engineer}
  {IPsoft}{London}{}
  {
    Client facing role writing requirements documents, then developing high impact automations. 
    Development is based on client requests as well as proactive suggestions based on experience 
    and analysis of event data using Jupyter.  All tasks were managed on JIRA sprint boards utilising 
    Bitbucket integration for git source control.\\~\\
    Automations primarily developed in JavaScript (backend) and Powershell, or Bash (client-side). 
    Where needed Python and Perl scripts were used to keep in line with client development frameworks.\\~\\
    Key automations:\\
    \begin{itemize}
      \item SCOM agent reporting: Query the SCOM APIs for agents in a downed state, on a schedule, 
            then rasing standard change requests via client systems, before restarting services on 
            the remote server. If required the agent would be upgraded. This would be followed up 
            with an HTML e-mail report to the product and server owners.\\ 
      \item VM provision Web Form: An automation that would generate a web form presenting options 
            to provision a VMware VM via PowerCli. Post-build scripts were deployed to VM to install 
            software requirements. This was written using JavaScript, HTML/CSS, PowerShell and Bash.\\
      \item Pre-application deployment: Multiple automations to stop application services across 
            multiple servers, before backing up config files (and databases), then calling external 
            deployments tools such as Jenkins. Finally restoring all services back to a running 
            state and validating their status.\\
    \end{itemize}
    Wrote and hosted the automation training courses for internal and client staff. This included 
    training with the IPcenter platform as well as an introduction to JavaScript, PowerShell and Bash.\\
  }

% IPsoft, Automation Engineer
\cventry % chktex 1
  {\mbox{May 2015 -} \mbox{June 2016}}
  {Automation Engineer}
  {IPsoft}{London}{}
  {
    Developing automations to trigger on events generated from alerts monitored by the IPcenter platform. 
    The automations developed were for Windows OS and application events.\\~\\
    Key automations:\\
    \begin{itemize}
      \item Service restarts: Restarting of failed services including analysis of system events and logs.\\
      \item Disk space: Analysis of local and remote disks, followed by file/folder deletion based on Regular Expression whitelist.\\
      \item Host down: Review of hosts not reporting to monitoring, in the case of virtual machine this 
            would include connecting to vCenter to perform additional check and restart of VM if required.\\
    \end{itemize}
    This development work was performed across multiple IPsoft clients, with the primary client requiring SC clearance.\\ 
  }

% Star Financial Systems, DevOps Engineer
\cventry % chktex 1
  {\mbox{May 2013 -} \mbox{May 2015}}
  {DevOps Engineer}
  {Star Financial Systems}{London}{}
  {
    Build and maintenance of the VMware hosted QA environments, as well as the datacenter, 
    hosted production servers. Provisioning of the QA builds was achieved using PowerShell - PowerCli 
    on Windows Server 2008R2 with TeamCity for application build/unit testing and OctopusDeploy 
    for application deployment (and MSSQL backups).\\~\\
    The use of Octopus Deploy along with PowerShell scripting helped to reduce the deployment time 
    to production from more than 6-hours to 30-minutes (30+ services distributed over 5 or more Windows hosts).\\~\\
    The use of PowerCli allowed for on-demand provisioning of standardized testing host for development and QA teams. 
    This helped to ensure that all client production environments were closely replicated in QA/dev builds.\\~\\
    All tools were integrated with Slack APIs to ensure real-time notifications for all teams.
    Including alert notifications from Nagios monitoring for production client environments.\\
  }

